\documentclass{article}%
\usepackage[T1]{fontenc}%
\usepackage[utf8]{inputenc}%
\usepackage{lmodern}%
\usepackage{textcomp}%
\usepackage{lastpage}%
\usepackage{graphicx}%
%
%
%
\begin{document}%
\normalsize%
\section{8{-}29{-}2017}%
\includegraphics[width=350px]{../lib/pictures/8-29-2017.jpg}%
\textbf{\newline%
\newline%
Explanation:\newline%
}%
    Why is Saturn partly blue?   The  featured picture  of Saturn approximates what a  human  would see if hovering close to the giant ringed world.      The image  was taken in 2006 March by the robot  Cassini spacecraft now orbiting  Saturn.    Here Saturn's majestic rings appear directly only as a thin vertical line.    The rings show their complex structure in the dark shadows they create on the image left.     Saturn's fountain moon Enceladus,  only about 500 kilometers across, is seen as the bump in the plane of the rings.    The northern hemisphere of  Saturn can appear partly blue for the same reason that  Earth's skies can appear blue {-}{-} molecules in the cloudless portions  of both planet's atmospheres are better at scattering blue light than red.    When looking deep into  Saturn's clouds, however, the natural  gold hue of Saturn's clouds becomes dominant.     It is not known why southern Saturn does not show the same blue hue {-}{-}  one hypothesis holds that clouds are higher there.      It is also  not known why Saturn's  clouds are colored gold.  Next month, Cassini will  end its mission with a final  dramatic dive into Saturn's atmosphere.

%
\end{document}